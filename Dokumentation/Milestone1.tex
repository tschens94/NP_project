\documentclass[a4paper]{article} %scrartcl
\usepackage[T1]{fontenc}
\usepackage[utf8]{inputenc}
\usepackage{fancyhdr}
\usepackage{vmargin}
\usepackage{multirow}	
\usepackage{amsmath}
\usepackage{amssymb}
\usepackage{comment}
\usepackage{graphics}
\usepackage{graphicx}
\usepackage{color}
\usepackage{framed}
\usepackage{epsfig}
\usepackage{subfigure}
\usepackage{tikz}
\usepackage{hyperref}
\usepackage{titlesec}
\usetikzlibrary{calc,trees,positioning,arrows,fit,shapes,calc}

\usepackage{listings}
\lstset{language=SQL,breaklines=true, keywordstyle = \color{blue},morekeywords={REPLACE, PARTITION, OVER, RANK, SUM, AVG}, breakatwhitespace=false,escapeinside={\%*}{*)}}

\newcommand*\bl{\big(}
\newcommand*\br{\big)}
\newcommand*\Bl{\Big(}
\newcommand*\Br{\Big)}
\newcommand*\bgl{\bigg(}
\newcommand*\bgr{\bigg)}
\newcommand*\Bgl{\Bigg(}
\newcommand*\Bgr{\Bigg)}
\newcommand*\closeB{)\br\Br}
\newcommand*\closebg{\closeB\bgr}
\newcommand*\closeBg{\closebg\Bgr}


\DeclareMathOperator*{\btie}{\bowtie}

\def\union{\mathbin{\bigcup}}
\def\selection{\mathbin{\sigma}}
\def\projection{\mathbin{\pi}}
\def\difference{\mathbin{-}}
\def\crossproduct{\mathbin{\times}}
\def\rename{\mathbin{\rho}}
\def\intersect{\mathbin{\bigcap}}

\def\leftsemijoin{\mathbin{\ltimes}}
\def\rightsemijoin{\mathbin{\rtimes}}
\def\leftantisemijoin{\mathbin{\vartriangleright}}
\def\rightantisemijoin{\mathbin{\vartriangleleft}}

\def\grouping{\mathbin{\gamma}}

\def\join{\mathbin{\bowtie}}
\def\ojoin{\setbox0=\hbox{$\bowtie$}%
  \rule[-.02ex]{.25em}{.4pt}\llap{\rule[\ht0]{.25em}{.4pt}}}
\def\leftouterjoin{\mathbin{\ojoin\mkern-5.8mu\bowtie}}
\def\rightouterjoin{\mathbin{\bowtie\mkern-5.8mu\ojoin}}
\def\fullouterjoin{\mathbin{\ojoin\mkern-5.8mu\bowtie\mkern-5.8mu\ojoin}}

\setlength{\parindent}{0pt}
\setlength{\parskip}{5pt}
\setlength{\headheight}{23pt}
\setlength{\headsep}{43pt}

\frenchspacing
\pagestyle{fancy}
\sloppy 

\markright{headline}

%Makro für das Relationalenmodell
\newcommand{\relation}[2]{[#1]: \{[#2]\}}

%Umgebung für verschiedene Aufgaben
\newenvironment{task}[2]{\subsection*{Aufgabe #1 #2}}{}

%Umgebung für verschiedene Sections
\newenvironment{sect}[2]{\section{#1 #2}}{}
\usepackage{pdfpages}
\usepackage{listings}

\begin{document}

% Add the title of the assignment sheet, your names and your Tutorialgroup here:
\newcommand{\subttl}{\textbf{Milestone 1}}
\newcommand{\StudNameOne}{Jens Heinen (2542182)}
\newcommand{\StudNameTwo}{Lukas Schaal (2539218)}
\newcommand{\StudNameThree}{Christoph Rosenhauer (2549220)}
\lhead{\begin{tabular}{l}
Nebenläufige Programmierung SoSe 2015 \\Universität des Saarlandes\\
\end{tabular}}

\rhead{\begin{tabular}{rc} Studenten: \StudNameOne & \\
		\qquad \StudNameTwo \\ 
		\qquad \StudNameThree \\
		\subtitle 
	\end{tabular}}
	
	\author{\StudNumeOne \& \StudNameTwo \& \StudNameThree}

\newcommand{\ndt}{Nichtdeterminismus }
Test test test

\begin{itemize}
	\item Anzahl der Threads geregelt über von außern steuerbaren Parameter
	\item Graphen in unterschiedliche Klassen unterteilen
	\item Objekte werden in Oberklasse zusammengefasst 
	\item MP oder SM ? noch zu klären
	\item Notiz: Max Wert von Initialisierung wird nie überschritten!
	\item Unterteilung der Graphstrukur in gröbere "Spalten", die aus mehreren Einzelspalten bestehen
	\item Jede Spalte wird von einem Thread bearbeitet
	
\end{itemize}


\section{Allgemeines \& Überblick}
Zur Lösung des nebenläufigen Problems wird die Graphenstruktur insgesamt in m gröbere Spalten unterteilt, wobei m als steuerbarer Parameter implementiert werden kann. Zusätz\-lich werden sog. Synchronisationsgrenzen implementiert, die angeben, nach wie vielen Iterationsschritten die Akkumulatoren an den Spaltenübergängen synchronisiert werden. Dieser Parameter soll ebenfalls von außen steuerbar sein. 
%TODO klassennamen einfügen 
Klassenstruktur: 
Ein Graph wird dynamisch aufgebaut. Dafür gibt es eine Klasse \textit{klasse}. Ein Knoten entsteht, sobald sein Wert positiv wird. Die Knoten werden in einer Oberklasse \textit{Oberklasse} zusammengesetzt. Zur spaltenweise Abarbeitung wird es eine Klasse \textit{Column?} geben. 

\section{Nebenläufige Probleme}
Nebenläufige Probleme treten vor allem bei der iterativen Berechnung der Spalten durch mehrere Threads, als auch der Synchronisation der Akkumulatoren auf. Dazu gehen wir folgenden Weg:
\begin{itemize}
	\item Die Synchronisation soll 
\end{itemize}
%%%%%%%%%%%%%Here-comes-your-document%%%%%%%%%%%%
%%%%%%%%%%%%%%%%%%%%%%%%%%%%%%%%%%%%%%%%%%%%%%%%

\end{document}

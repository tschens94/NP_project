\documentclass[a4paper]{article} %scrartcl
\usepackage[T1]{fontenc}
\usepackage[utf8]{inputenc}
\usepackage{fancyhdr}
\usepackage{vmargin}
\usepackage{multirow}	
\usepackage{amsmath}
\usepackage{amssymb}
\usepackage{comment}
\usepackage{graphics}
\usepackage{graphicx}
\usepackage{color}
\usepackage{framed}
\usepackage{epsfig}
\usepackage{subfigure}
\usepackage{tikz}
\usepackage{hyperref}
\usetikzlibrary{calc,trees,positioning,arrows,fit,shapes,calc}

\usepackage{listings}
\lstset{language=SQL,breaklines=true, keywordstyle = \color{blue},morekeywords={REPLACE, PARTITION, OVER, RANK, SUM, AVG}, breakatwhitespace=false,escapeinside={\%*}{*)}}

\newcommand*\bl{\big(}
\newcommand*\br{\big)}
\newcommand*\Bl{\Big(}
\newcommand*\Br{\Big)}
\newcommand*\bgl{\bigg(}
\newcommand*\bgr{\bigg)}
\newcommand*\Bgl{\Bigg(}
\newcommand*\Bgr{\Bigg)}
\newcommand*\closeB{)\br\Br}
\newcommand*\closebg{\closeB\bgr}
\newcommand*\closeBg{\closebg\Bgr}


\DeclareMathOperator*{\btie}{\bowtie}

\def\union{\mathbin{\bigcup}}
\def\selection{\mathbin{\sigma}}
\def\projection{\mathbin{\pi}}
\def\difference{\mathbin{-}}
\def\crossproduct{\mathbin{\times}}
\def\rename{\mathbin{\rho}}
\def\intersect{\mathbin{\bigcap}}

\def\leftsemijoin{\mathbin{\ltimes}}
\def\rightsemijoin{\mathbin{\rtimes}}
\def\leftantisemijoin{\mathbin{\vartriangleright}}
\def\rightantisemijoin{\mathbin{\vartriangleleft}}

\def\grouping{\mathbin{\gamma}}

\def\join{\mathbin{\bowtie}}
\def\ojoin{\setbox0=\hbox{$\bowtie$}%
  \rule[-.02ex]{.25em}{.4pt}\llap{\rule[\ht0]{.25em}{.4pt}}}
\def\leftouterjoin{\mathbin{\ojoin\mkern-5.8mu\bowtie}}
\def\rightouterjoin{\mathbin{\bowtie\mkern-5.8mu\ojoin}}
\def\fullouterjoin{\mathbin{\ojoin\mkern-5.8mu\bowtie\mkern-5.8mu\ojoin}}

\setlength{\parindent}{0pt}
\setlength{\parskip}{5pt}
\setlength{\headheight}{23pt}
\setlength{\headsep}{43pt}

\frenchspacing
\pagestyle{fancy}
\sloppy 

\markright{headline}

%Makro für das Relationalenmodell
\newcommand{\relation}[2]{[#1]: \{[#2]\}}

%Umgebung für verschiedene Aufgaben
\newenvironment{task}[2]{\subsection*{Aufgabe #1 #2}}{}

%Umgebung für verschiedene Sections
\newenvironment{sect}[2]{\section{#1 #2}}{}
\usepackage{pdfpages}
\usepackage{listings}

\begin{document}

% Add the title of the assignment sheet, your names and your Tutorialgroup here:
\newcommand{\subttl}{\textbf{Milestone 1}}
\newcommand{\StudNameOne}{Christoph Rosenhauer (2549220)}
\newcommand{\StudNameTwo}{Lukas Schaal (2539218)}
\newcommand{\StudNameThree}{Jens Heinen (2542182)}
\lhead{\begin{tabular}{l}
Nebenläufige Programmierung SoSe 2015 \\Universität des Saarlandes\\
\end{tabular}}

\rhead{\begin{tabular}{rc} Studenten: \StudNameOne & \\
		\qquad \StudNameTwo \\ 
		\qquad \StudNameThree \\
		\subtitle 
	\end{tabular}}
	
	\author{\StudNumeOne \& \StudNameTwo \& \StudNameThree}


\section{Allgemeines \& Überblick}
%TODO verbessern
Die Graphstruktur $n \times m$ soll durch eine variable Anzahl von Threads bearbeitet werden. Dabei kann das von 1 Thread (optional sequentielles Bearbeiten) bis zu m Threads reichen, so dass jede Spalte des Graphs in jeder Iteration von genau einem Thread bearbeitet werden. 
Zusätzlich werden sog. Synchronisationsgrenzen  angegeben, nach wie vielen Iterationsschritten die Akkumulatoren an den Spaltenübergängen auf jeden synchronisiert werden sollen, unabhängig von lokaler Konvergenz. Dieser Parameter soll von außen steuerbar sein. 

\section{Nebenläufige Probleme}
Hier gehen wir folgenden Weg:
Der Zeitpunkt der Synchronisation soll anhand eines Parameters erfolgen. Falls diese Bedingung erfüllt ist, werden mitels Shared Memory die Daten der Akkumulatoren ausgetauscht, wie folgt: 

Nach x festgelegten Iterationstasks wird ein Synchronisationstask an den Threadpool übergeben. An dieser Stelle ist wichtig, dass zum Beginn der Synchronisation zweier (oder mehrerer Spalten) kein Thread mehr einen Iterationstask auf einer der Spalten ausführt. Ein Thread nimmt sich nun die Referenzen auf die Nachbarspalten und verrechnet Akkumulatordaten und aktuelle Werte der Knoten. Hierbei muss darauf geachtet werden, ob die propagierenden Spalten von einem oder mehreren Threads bearbeitet werden. In Abhängigkeit 

\section{Verwendete Datenstrukur}
Der gegebene Graph ist ein $n\times m$ Graph. Dabei wird das Objekt intern auch in m Spalten unterteilt. 

Die Anzahl der operierenden Threads wird über einen Parameter gegeben. Ein Thread bekommt dabei einen Task übergeben und bearbeitet nach diesem Task eine oder mehrere Spalten des Graphen. Die Threads werden mittels eines Threadpools / Scheduler den Spalten zugewiesen und bei Bedarf erstellt. Die Tasks beinhalten lokalen Austausch, Synchronisation zwischen mehreren Spalten und globale Konvergenz.

Jede Spalte besteht aus ihren Knoten und Vektoren der links- und rechtsbenachbarten Akkumulatoren. Durch Call-by-Reference kann ein Thread, der gerade Spalte $j$ bearbeitet, auch Daten von Spalte $j-1$ \& $j+1$ bekommen. An dieser Stelle sollte durch genaues Locking die Gefahr von Data Races verhindert werden. 

Ein Knoten ist ein Objekt, das nur seinen aktuellen Wert kennt. Die Übergangsrate ist in der GraphInfo Klasse enthalten. 

\section{Konvergenz}
\subsection{Lokale Konvergenz}
Die lokale Konvergenz innerhalb einer Spalte wird nach festgelegten Iterationsschritten (abhängig von Parameter) überprüft. Im Falle einer erkannten lokalen Konvergenz innerhalb einer Spalte wird mit den benachbarten Spalten synchronisiert. Dafür werden der Iterationszustand i von Spalte j und  Zustand k von Spalte j+1 angepasst, so dass i = k gilt. Dazu wird die Spalte mit der lokalen Konvergenz so lange iteriert, bis sie auf dem gleichen Stand ist wie die benachbarte Spalte. Falls nicht, wird nach x Iterationen definitiv synchronisiert, wobei x von einem Parameter abgeleitet wird. Inwiefern x davon abhängt, wird durch Tests ermittelt und im Laufe der Implementierung optimiert. 

\subsection{Globale Konvergenz}
Globale Konvergenz erfolgt implizit durch lokale Konvergenz. Dazu existieren 2 Synchronisationsarten: Eine festgelegt durch den Parameter, eine bei lokaler Konvergenz einer Spalte.
Wenn lokale Konvergenz vorliegt, wird das Verhältnis zwischen Inflow und Outflow der betreffenden Spalten überprüft. Wenn der  Unterschied zwischen Inflow \& Outflow zu groß wird , muss in kleineren Iterationsdistanzen überprüft werden. 
%Als Datenstruktur implementieren wir eine \textit{ArrayList?} deren Elemente einzelne Knoten sind. Jeder Knoten speichert linke und rechte Akkumulatorenfelder.  
%%%%%%%%%%%%%Here-comes-your-document%%%%%%%%%%%%
%%%%%%%%%%%%%%%%%%%%%%%%%%%%%%%%%%%%%%%%%%%%%%%%

\end{document}
